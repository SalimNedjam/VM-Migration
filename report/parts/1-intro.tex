\section*{Remerciements}

Je tiens à remercier les personnes travaillant au sein de l'équipe R\&D de GANDI, pour leur sympathie et leur disponibilité.
Mes remerciements s’adressent tout particulièrement à mon responsable de stage, Antoine Blin, je lui témoigne ici toute ma gratitude pour m'avoir guidé durant ces mois.


Je remercie Pierre Sens, qui a patiemment répondu à mes questions concernant ce stage.


Je souhaite remercier mon frère et ma sœur qui ont été toujours là pour moi et qui me servent de modèles dans ma vie.


Je souhaite exprimer toute ma gratitude et reconnaissance à mes parents, mes mots ne seraient jamais à la hauteur de l’amour et l’affection que vous m’avez témoignée tout au long de ma vie.
Cette dédicace serait pour moi, la meilleure façon de vous honorer et de vous montrer à quel point vous avez été magnifique, sans vous, je ne serrai jamais arrivé où je suis aujourd'hui. 

\newpage

\pagenumbering{arabic}
\setcounter{page}{1}
\section*{Introduction}
Actuellement, en dernière année de Master Systèmes et Applications Répartis à Sorbonne Université Science, le dernier semestre de mon cursus scolaire est consacré à la réalisation d’un stage au sein d’une société.
Il s’est déroulé du 08 mars au 08 septembre 2021 chez l’entreprise GANDI au sein de l'équipe Recherche et Développement.



Au fil des années, Linux est devenu le premier choix pour le développement de solutions sur le cloud.
De nombreux fournisseurs de clouds publics prospères utilisent la virtualisation Linux pour alimenter leurs infrastructures sous-jacente.
La consolidation et répartition des charges deviennent des éléments importants pour améliorer les capacités ces infrastructures complexes.
La migration de machines virtuelles est une des technique pour répondre à cette problématique.
Ces déplacements des VMs doivent se faire sans interruption de service et dans des délais très réduits afin de respecter la qualité globale des services.
Ces migrations imposent dès lors des responsabilités quant à l'intégrité des données transférées.
En effet, certaines précautions doivent être prises en amont pour qu’aucune incompatibilité ne survienne entre les machines concernées par le transfert.
Ce stage s’inscrit dans cette dynamique et cherche à sécuriser les mécanismes de migration des machines virtuelles afin de préserver l'intégrité des données migrées.


Nous allons maintenant aborder l’environnement technique au sein duquel mon travail a été effectué.
Nous verrons notamment les avantages de la virtualisation et détaillerons les principes de la migration des machines virtuelles.
Ensuite, nous aborderons les techniques de migration des machines virtuelles et décrirons les principales structures de données utilisées.
Nous parlerons par la suite du travail proposée, les problèmes rencontrés et les solutions apportées.
Enfin, nous allons évaluer le travail avec différents benchmarks.