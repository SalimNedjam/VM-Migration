\newpage
\section*{Conclusion}
L’arrivée de la migration de machines virtuelles a rajouté une véritable plus value en ce qui concerne les services hébergés sur le cloud.
En effet, nous pouvons atteindre des durée d'uptime jamais vu auparavant, impactant directement la qualité de service.
Cependant, ces techniques ne sont pas sans failles pouvant altérer l'intégrité des données des machines virtuelles.

Nous avons pu, en utilisant la génération d'empreintes de machines virtuelles, garantir l'intégrité des données transférées à travers le réseau et ainsi éviter une l'altération de la machine pendant la migration.
Il a fallu prendre en compte le temps de génération d'empreintes pour ne pas dégrader le QOS ainsi que la bande passante réseau.

Ce stage a été très enrichissant à plusieurs points de vue.
J’ai pu, d’un point de vue interne, découvrir le fonctionnement d’un service de recherche et développement, et appris comment appréhender un problème complexe.
De plus, les nombreux problèmes auxquels j'ai été confronté m'ont demandé une certaine rigueur pour les résoudre.
J'ai également eu la possibilité d'étudier le fonctionnement interne de la migration des machines virtuelles, j'ai ainsi pu approfondir mes connaissances techniques dans le domaine de la virtualisation en général, mais aussi comment garantir l'intégrité d'une machine virtuelle migrée.
Nous pouvons imaginer différentes pistes pour optimiser l'overhead dû à la génération de l'empreinte de la machine virtuelle avec une gestion plus intelligente des pages de mémoire vide et dirty, ainsi que la prise en charge du multi-threading.